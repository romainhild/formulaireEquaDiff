\documentclass{article}

\usepackage{amsmath,amssymb}
\usepackage[
  left=50pt,
  right=50pt,
  top=50pt,
  bottom=50pt]{geometry}

\begin{document}
\section{Operators}
All scalar functions are noted as regular letter, as $f$, all vector field are
noted as bold letter, as $\mathbf{f}=(f_1,\dots,f_d)$, and tensor field with two
bars upon them, as $\overline{\overline{f}}=(f_{ij})_{i,j=1,\dots,d}$.\\
The space variable can be written $\mathbf{x}=(x_1,\dots,x_d)=(x,y,z)$.

\begin{itemize}
\item
  Scalar product of two vectorial fields:
  \begin{equation*}
    \mathbf{u}\cdot\mathbf{v} = \sum_{i=1}^d u_i v_i
  \end{equation*}
\item
  Scalar product of two tensor fields:
  \begin{equation*}
    \overline{\overline{m}}:\overline{\overline{n}} =
    tr(\overline{\overline{m}}*\overline{\overline{n}}^T)
  \end{equation*}
  where $tr$ is the trace of the tensor, and $\overline{\overline{n}}^T$ is the
  transpose of the tensor.
\item
  Gradient of a scalar field:
  \begin{equation*}
    \nabla f =
    \begin{pmatrix}
      \frac{\partial f}{\partial x_1}
      & \dots
      & \frac{\partial f}{\partial x_d}
    \end{pmatrix}
  \end{equation*}
\item
  Gradient of a vectorial field:
  \begin{equation*}
    \nabla \mathbf{f} =
    \begin{pmatrix}
      \frac{\partial f_1}{\partial x_1} &
      \dots &
      \frac{\partial f_1}{\partial x_d} \\
      \vdots & \ddots & \vdots \\
      \frac{\partial f_d}{\partial x_1} &
      \dots &
      \frac{\partial f_d}{\partial x_d}
    \end{pmatrix}
  \end{equation*}
\item
  Divergence of a vectorial field:
  \begin{equation*}
    \nabla\cdot\mathbf{f} =
    \sum_{i=1}^d \frac{\partial f_i}{\partial x_i}
  \end{equation*}
\item
  Curl of a vectorial field:
  \begin{equation*}
    \nabla\times\mathbf{f} =
    \begin{pmatrix}
      \frac{\partial f_3}{\partial x_2} - \frac{\partial f_2}{\partial x_3} \\
      \frac{\partial f_1}{\partial x_3} - \frac{\partial f_3}{\partial x_1} \\
      \frac{\partial f_2}{\partial x_1} - \frac{\partial f_1}{\partial x_2}
    \end{pmatrix}
  \end{equation*}
  In 2D, we only keep the last component as a scalar field.
\item
  Laplacian of a scalar field:
  \begin{equation*}
    \Delta f = \nabla\cdot\nabla f =
    \sum_{i=1}^d \frac{\partial^2 f}{\partial x_i^2}
  \end{equation*}
\item
  Laplacian of a vectorial field:
  \begin{equation*}
    \Delta \mathbf{f} =
    \begin{pmatrix}
      \Delta f_1 \\ \vdots \\ \Delta f_d
    \end{pmatrix} =
    \begin{pmatrix}
      \sum_{i=1}^d \frac{\partial^2 f_1}{\partial x_i^2}\\
      \vdots \\
      \sum_{i=1}^d \frac{\partial^2 f_d}{\partial x_i^2}
    \end{pmatrix}
  \end{equation*}
\item
  Directional derivatives along a vector $\mathbf{v}$ of a scalar field $f$:
  \begin{equation*}
    (\mathbf{v}\cdot\nabla)f =
    \sum_{i=1}^d v_i\frac{\partial f}{\partial x_i}
  \end{equation*}
\item
  Directional derivatives along a vector $\mathbf{v}$ of a vectorial field
  $\mathbf{u}$:
  \begin{equation*}
    (\mathbf{v}\cdot\nabla)\mathbf{u} =
    \begin{pmatrix}
      (\mathbf{v}\cdot\nabla)u_1 \\
      \vdots \\
      (\mathbf{v}\cdot\nabla)u_d
    \end{pmatrix} =
    \begin{pmatrix}
      \sum_{i=1}^d v_i\frac{\partial u_1}{\partial x_i} \\
      \vdots \\
      \sum_{i=1}^d v_i\frac{\partial u_d}{\partial x_i}
    \end{pmatrix}
  \end{equation*}
\end{itemize}

\section{Vectorial identities}
\subsection{Addition and Multiplication}
\begin{itemize}
\item
  $\mathbf{A}+\mathbf{B}=\mathbf{B}+\mathbf{A}$
\item
  $\mathbf{A}\cdot\mathbf{B}=\mathbf{B}\cdot\mathbf{A}$
\item
  $\mathbf{A}\times\mathbf{B}=\mathbf{-B}\times\mathbf{A}$
\item
  $\left(\mathbf{A}+\mathbf{B}\right)\cdot\mathbf{C}=
  \mathbf{A}\cdot\mathbf{C}+\mathbf{B}\cdot\mathbf{C}$
\item
  $\left(\mathbf{A}+\mathbf{B}\right)\times\mathbf{C}=
  \mathbf{A}\times\mathbf{C}+\mathbf{B}\times\mathbf{C}$
\item
  $\mathbf{A}\cdot\left(\mathbf{B}\times\mathbf{C}\right)=
  \mathbf{B}\cdot\left(\mathbf{C}\times\mathbf{A}\right)=
  \mathbf{C}\cdot\left(\mathbf{A}\times\mathbf{B}\right)$
\item
  $\mathbf{A}\times\left(\mathbf{B}\times\mathbf{C}\right)=
  \left(\mathbf{A}\cdot\mathbf{C}\right)\mathbf{B}
  -\left(\mathbf{A}\cdot\mathbf{B}\right)\mathbf{C}$
\item
  $\left(\mathbf{A}\times\mathbf{B}\right)\times\mathbf{C}=
  \left(\mathbf{A}\cdot\mathbf{C}\right)\mathbf{B}
  -\left(\mathbf{B}\cdot\mathbf{C}\right)\mathbf{A}$
\item
  $\left(\mathbf{A}\times\mathbf{B}\right)
  \cdot\left(\mathbf{C}\times\mathbf{D}\right)=
  \left(\mathbf{A}\cdot\mathbf{C}\right)\left(\mathbf{B}\cdot\mathbf{D}\right)
  -\left(\mathbf{B}\cdot\mathbf{C}\right)\left(\mathbf{A}\cdot\mathbf{D}\right)$
\item
  $\left(\mathbf{A}\cdot\left(\mathbf{B}\times\mathbf{C}\right)\right)\mathbf{D}
  =\left(\mathbf{A}\cdot\mathbf{D}\right)\left(\mathbf{B}\times\mathbf{C}\right)
  +\left(\mathbf{B}\cdot\mathbf{D}\right)\left(\mathbf{C}\times\mathbf{A}\right)
  +\left(\mathbf{C}\cdot\mathbf{D}\right)\left(\mathbf{A}\times\mathbf{B}\right)
  $
\item
  $\left(\mathbf{A}\times\mathbf{B}\right)
  \times\left(\mathbf{C}\times\mathbf{D}\right)
  =\left(\mathbf{A}\cdot\left(\mathbf{B}\times\mathbf{D}\right)\right)\mathbf{C}
  -\left(\mathbf{A}\cdot\left(\mathbf{B}\times\mathbf{C}\right)\right)\mathbf{D}
  $
\end{itemize}

\subsection{Differentiation}
\subsubsection{Gradient}
\begin{itemize}
\item
  $\nabla(\psi+\phi)=\nabla\psi+\nabla\phi $
\item
  $\nabla (\psi \, \phi) = \phi \,\nabla \psi  + \psi \,\nabla \phi $
\item
  $\nabla\left(\mathbf{A}\cdot\mathbf{B}\right)=
  \left(\mathbf{A}\cdot\nabla\right)\mathbf{B}
  +\left(\mathbf{B}\cdot\nabla\right)\mathbf{A}
  +\mathbf{A}\times\left(\nabla\times\mathbf{B}\right)
  +\mathbf{B}\times\left(\nabla\times\mathbf{A}\right) $
\end{itemize}

\subsubsection{Divergence}
\begin{itemize}
\item
  $\nabla\cdot(\mathbf{A}+\mathbf{B})=
  \nabla\cdot\mathbf{A}+\nabla\cdot\mathbf{B}$
\item
  $\nabla\cdot\left(\psi\mathbf{A}\right)=
  \psi\nabla\cdot\mathbf{A}+\mathbf{A}\cdot\nabla \psi $
\item
  $\nabla\cdot\left(\mathbf{A}\times\mathbf{B}\right)=
  \mathbf{B}\cdot(\nabla\times\mathbf{A})
  -\mathbf{A}\cdot(\nabla\times\mathbf{B})$
\end{itemize}

\subsubsection{Curl}
\begin{itemize}
\item
  $\nabla\times(\mathbf{A}+\mathbf{B})=
  \nabla\times\mathbf{A}+\nabla\times\mathbf{B} $
\item
  $\nabla\times\left(\psi\mathbf{A}\right)=
  \psi\nabla\times\mathbf{A}+\nabla\psi\times\mathbf{A}$
\item
  $\nabla\times\left(\mathbf{A}\times\mathbf{B}\right)=
  \mathbf{A}\left(\nabla\cdot\mathbf{B}\right)
  -\mathbf{B}\left(\nabla\cdot\mathbf{A}\right)
  +\left(\mathbf{B}\cdot\nabla\right)\mathbf{A}
  -\left(\mathbf{A}\cdot\nabla\right)\mathbf{B} $
\end{itemize}

\subsubsection{Second derivatives}
\begin{itemize}
\item
  $\nabla\cdot(\nabla\times\mathbf{A})=0 $
\item
  $\nabla\times(\nabla\psi)= \mathbf{0} $
\item
  $\nabla\cdot(\nabla\psi)=\nabla^{2}\psi $
\item
  $\nabla\left(\nabla\cdot\mathbf{A}\right)
  -\nabla\times\left(\nabla\times\mathbf{A}\right)=\nabla^{2}\mathbf{A} $
\item
  $\nabla\cdot(\phi\nabla\psi)=\phi\nabla^{2}\psi + \nabla\phi\cdot\nabla\psi $
\item
  $\psi\nabla^2\phi-\phi\nabla^2\psi=
  \nabla\cdot\left(\psi\nabla\phi-\phi\nabla\psi\right)$
\item
  $\nabla^2(\phi\psi)=
  \phi\nabla^2\psi+2\nabla\phi\cdot\nabla\psi+\psi\nabla^2\phi$
\item
  $\nabla^2(\psi\mathbf{A})=
  \mathbf{A}\nabla^2\psi+2(\nabla\psi\cdot\nabla)\mathbf{A}
  +\psi\nabla^2\mathbf{A}$
\item
  $\nabla^2(\mathbf{A}\cdot\mathbf{B})=
  \mathbf{A}\cdot\nabla^2\mathbf{B} - \mathbf{B}\cdot\nabla^2\mathbf{A}
  + 2\nabla\cdot((\mathbf{B}\cdot\nabla)\mathbf{A}
  + \mathbf{B}\times\nabla\times\mathbf{A})$
\end{itemize}

\subsubsection{Third derivatives}
\begin{itemize}
\item
  $\nabla^{2}(\nabla\psi) = \nabla(\nabla\cdot(\nabla\psi))
  = \nabla(\nabla^{2}\psi)$
\item
  $\nabla^{2}(\nabla\cdot\mathbf{A})
  =\nabla\cdot(\nabla(\nabla\cdot\mathbf{A}))=\nabla\cdot(\nabla^{2}\mathbf{A})$
\item
  $\nabla^{2}(\nabla\times\mathbf{A}) =
  -\nabla\times(\nabla\times(\nabla\times\mathbf{A}))
  = \nabla\times(\nabla^{2}\mathbf{A})$
\end{itemize}

\subsection{Integration}
\subsubsection{Surface–volume integrals}
In the following surface–volume integral theorems, $V$ denotes a 3d volume with
a corresponding 2d boundary $S = \partial V$. And
$\mathbf{A}\cdot d\mathbf{S}=\mathbf{A}\cdot\mathbf{n}\, dS$.
\begin{itemize}
\item
  $\int_{\partial V} \mathbf{A}\cdot d\mathbf{S}=
  \int_V \left(\nabla\cdot\mathbf{A}\right)dV$ (Divergence theorem)
\item
  $\int_{\partial V}\psi d \mathbf{S} = \int_V \nabla \psi\, dV$
\item
  $\int_{\partial V}\left(\hat{\mathbf{n}}\times\mathbf{A}\right)dS=
  \int _{V}\left(\nabla\times\mathbf{A}\right)dV$
\item
  $\int_{\partial V}\psi\left(\nabla\varphi\cdot\hat{\mathbf{n}}\right)dS =
  \int _{V}\left(\psi\nabla^{2}\varphi+\nabla\varphi\cdot\nabla\psi\right)dV$
  (Green's first identity)
\item
  $\int_{\partial V}\left[\left(\psi\nabla\varphi
  -\varphi\nabla\psi\right)\cdot\hat{\mathbf{n}}\right]dS=
  \int_{\partial V}\left[\psi\frac{\partial\varphi}{\partial n}
    -\varphi\frac{\partial\psi}{\partial n}\right]dS
  =\int_{V}\left(\psi\nabla^{2}\varphi-\varphi\nabla^{2}\psi\right)dV$
  (Green's second identity)
\end{itemize}

\subsubsection{Curve–surface integrals}
In the following curve–surface integral theorems, $S$ denotes a 2d open surface
with a corresponding 1d boundary $C = \partial S$:
\begin{itemize}
\item
  $\int_{\partial S}\mathbf{A}\cdot d\boldsymbol{\ell}=
  \int_{S}\left(\nabla\times\mathbf{A}\right)\cdot d\mathbf{S}$
  (Stokes' theorem)
\item
  $\int_{\partial S}\psi d\boldsymbol{\ell}
  =\int_{S}\left(\hat{\mathbf{n}}\times\nabla\psi\right)dS$
\end{itemize}

\section{Spaces}
$L_2,W,H,H(\mathrm{div,curl})$, De Rham, inequality between norms, Cauchy-Schwartz...

\section{Finite elements}

\section{Equations}
\subsection{Stokes}
\subsection{Navier-Stokes}
\subsection{Heat}
\subsection{Elasticity}
\subsection{Maxwell}

\end{document}
