\documentclass{article}

\usepackage[utf8]{inputenc}
\usepackage[T1]{fontenc}
\usepackage[english]{babel}
\usepackage{amsmath,amssymb}
\usepackage[
  left=50pt,
  right=50pt,
  top=50pt,
  bottom=50pt]{geometry}
\usepackage{float}
\usepackage{array}

\newcommand{\bE}{\mathbf{E}}
\newcommand{\bH}{\mathbf{H}}
\newcommand{\bD}{\mathbf{D}}
\newcommand{\bB}{\mathbf{B}}
\newcommand{\bj}{\mathbf{j}}
\newcommand{\bA}{\mathbf{A}}
\newcommand{\bv}{\mathbf{v}}

\begin{document}
\section{Operators}
All scalar functions are noted as regular letter, as $f$, all vector field are
noted as bold letter, as $\mathbf{f}=(f_1,\dots,f_d)$, and tensor field with two
bars upon them, as $\overline{\overline{f}}=(f_{ij})_{i,j=1,\dots,d}$.\\
The space variable can be written $\mathbf{x}=(x_1,\dots,x_d)=(x,y,z)$.

\begin{itemize}
\item
  Scalar product of two vectorial fields:
  \begin{equation*}
    \mathbf{u}\cdot\mathbf{v} = \sum_{i=1}^d u_i v_i
  \end{equation*}
\item
  Scalar product of two tensor fields:
  \begin{equation*}
    \overline{\overline{m}}:\overline{\overline{n}} =
    tr(\overline{\overline{m}}*\overline{\overline{n}}^T)
  \end{equation*}
  where $tr$ is the trace of the tensor, and $\overline{\overline{n}}^T$ is the
  transpose of the tensor.
\item
  Gradient of a scalar field:
  \begin{equation*}
    \nabla f =
    \begin{pmatrix}
      \frac{\partial f}{\partial x_1}
      & \dots
      & \frac{\partial f}{\partial x_d}
    \end{pmatrix}
  \end{equation*}
\item
  Gradient of a vectorial field:
  \begin{equation*}
    \nabla \mathbf{f} =
    \begin{pmatrix}
      \frac{\partial f_1}{\partial x_1} &
      \dots &
      \frac{\partial f_1}{\partial x_d} \\
      \vdots & \ddots & \vdots \\
      \frac{\partial f_d}{\partial x_1} &
      \dots &
      \frac{\partial f_d}{\partial x_d}
    \end{pmatrix}
  \end{equation*}
\item
  Divergence of a vectorial field:
  \begin{equation*}
    \nabla\cdot\mathbf{f} =
    \sum_{i=1}^d \frac{\partial f_i}{\partial x_i}
  \end{equation*}
\item
  Curl of a vectorial field:
  \begin{equation*}
    \nabla\times\mathbf{f} =
    \begin{pmatrix}
      \frac{\partial f_3}{\partial x_2} - \frac{\partial f_2}{\partial x_3} \\
      \frac{\partial f_1}{\partial x_3} - \frac{\partial f_3}{\partial x_1} \\
      \frac{\partial f_2}{\partial x_1} - \frac{\partial f_1}{\partial x_2}
    \end{pmatrix}
  \end{equation*}
  In 2D, we only keep the last component as a scalar field.
\item
  Laplacian of a scalar field:
  \begin{equation*}
    \Delta f = \nabla\cdot\nabla f =
    \sum_{i=1}^d \frac{\partial^2 f}{\partial x_i^2}
  \end{equation*}
\item
  Laplacian of a vectorial field:
  \begin{equation*}
    \Delta \mathbf{f} =
    \begin{pmatrix}
      \Delta f_1 \\ \vdots \\ \Delta f_d
    \end{pmatrix} =
    \begin{pmatrix}
      \sum_{i=1}^d \frac{\partial^2 f_1}{\partial x_i^2}\\
      \vdots \\
      \sum_{i=1}^d \frac{\partial^2 f_d}{\partial x_i^2}
    \end{pmatrix}
  \end{equation*}
\item
  Directional derivatives along a vector $\mathbf{v}$ of a scalar field $f$:
  \begin{equation*}
    (\mathbf{v}\cdot\nabla)f =
    \sum_{i=1}^d v_i\frac{\partial f}{\partial x_i}
  \end{equation*}
\item
  Directional derivatives along a vector $\mathbf{v}$ of a vectorial field
  $\mathbf{u}$:
  \begin{equation*}
    (\mathbf{v}\cdot\nabla)\mathbf{u} =
    \begin{pmatrix}
      (\mathbf{v}\cdot\nabla)u_1 \\
      \vdots \\
      (\mathbf{v}\cdot\nabla)u_d
    \end{pmatrix} =
    \begin{pmatrix}
      \sum_{i=1}^d v_i\frac{\partial u_1}{\partial x_i} \\
      \vdots \\
      \sum_{i=1}^d v_i\frac{\partial u_d}{\partial x_i}
    \end{pmatrix}
  \end{equation*}
\end{itemize}

\section{Vectorial identities}
\subsection{Addition and Multiplication}
\begin{itemize}
\item
  $\mathbf{A}+\mathbf{B}=\mathbf{B}+\mathbf{A}$
\item
  $\mathbf{A}\cdot\mathbf{B}=\mathbf{B}\cdot\mathbf{A}$
\item
  $\mathbf{A}\times\mathbf{B}=\mathbf{-B}\times\mathbf{A}$
\item
  $\left(\mathbf{A}+\mathbf{B}\right)\cdot\mathbf{C}=
  \mathbf{A}\cdot\mathbf{C}+\mathbf{B}\cdot\mathbf{C}$
\item
  $\left(\mathbf{A}+\mathbf{B}\right)\times\mathbf{C}=
  \mathbf{A}\times\mathbf{C}+\mathbf{B}\times\mathbf{C}$
\item
  $\mathbf{A}\cdot\left(\mathbf{B}\times\mathbf{C}\right)=
  \mathbf{B}\cdot\left(\mathbf{C}\times\mathbf{A}\right)=
  \mathbf{C}\cdot\left(\mathbf{A}\times\mathbf{B}\right)$
\item
  $\mathbf{A}\times\left(\mathbf{B}\times\mathbf{C}\right)=
  \left(\mathbf{A}\cdot\mathbf{C}\right)\mathbf{B}
  -\left(\mathbf{A}\cdot\mathbf{B}\right)\mathbf{C}$
\item
  $\left(\mathbf{A}\times\mathbf{B}\right)\times\mathbf{C}=
  \left(\mathbf{A}\cdot\mathbf{C}\right)\mathbf{B}
  -\left(\mathbf{B}\cdot\mathbf{C}\right)\mathbf{A}$
\item
  $\left(\mathbf{A}\times\mathbf{B}\right)
  \cdot\left(\mathbf{C}\times\mathbf{D}\right)=
  \left(\mathbf{A}\cdot\mathbf{C}\right)\left(\mathbf{B}\cdot\mathbf{D}\right)
  -\left(\mathbf{B}\cdot\mathbf{C}\right)\left(\mathbf{A}\cdot\mathbf{D}\right)$
\item
  $\left(\mathbf{A}\cdot\left(\mathbf{B}\times\mathbf{C}\right)\right)\mathbf{D}
  =\left(\mathbf{A}\cdot\mathbf{D}\right)\left(\mathbf{B}\times\mathbf{C}\right)
  +\left(\mathbf{B}\cdot\mathbf{D}\right)\left(\mathbf{C}\times\mathbf{A}\right)
  +\left(\mathbf{C}\cdot\mathbf{D}\right)\left(\mathbf{A}\times\mathbf{B}\right)
  $
\item
  $\left(\mathbf{A}\times\mathbf{B}\right)
  \times\left(\mathbf{C}\times\mathbf{D}\right)
  =\left(\mathbf{A}\cdot\left(\mathbf{B}\times\mathbf{D}\right)\right)\mathbf{C}
  -\left(\mathbf{A}\cdot\left(\mathbf{B}\times\mathbf{C}\right)\right)\mathbf{D}
  $
\end{itemize}

\subsection{Differentiation}
\subsubsection{Gradient}
\begin{itemize}
\item
  $\nabla(\psi+\phi)=\nabla\psi+\nabla\phi $
\item
  $\nabla (\psi \, \phi) = \phi \,\nabla \psi  + \psi \,\nabla \phi $
\item
  $\nabla\left(\mathbf{A}\cdot\mathbf{B}\right)=
  \left(\mathbf{A}\cdot\nabla\right)\mathbf{B}
  +\left(\mathbf{B}\cdot\nabla\right)\mathbf{A}
  +\mathbf{A}\times\left(\nabla\times\mathbf{B}\right)
  +\mathbf{B}\times\left(\nabla\times\mathbf{A}\right) $
\end{itemize}

\subsubsection{Divergence}
\begin{itemize}
\item
  $\nabla\cdot(\mathbf{A}+\mathbf{B})=
  \nabla\cdot\mathbf{A}+\nabla\cdot\mathbf{B}$
\item
  $\nabla\cdot\left(\psi\mathbf{A}\right)=
  \psi\nabla\cdot\mathbf{A}+\mathbf{A}\cdot\nabla \psi $
\item
  $\nabla\cdot\left(\mathbf{A}\times\mathbf{B}\right)=
  \mathbf{B}\cdot(\nabla\times\mathbf{A})
  -\mathbf{A}\cdot(\nabla\times\mathbf{B})$
\end{itemize}

\subsubsection{Curl}
\begin{itemize}
\item
  $\nabla\times(\mathbf{A}+\mathbf{B})=
  \nabla\times\mathbf{A}+\nabla\times\mathbf{B} $
\item
  $\nabla\times\left(\psi\mathbf{A}\right)=
  \psi\nabla\times\mathbf{A}+\nabla\psi\times\mathbf{A}$
\item
  $\nabla\times\left(\mathbf{A}\times\mathbf{B}\right)=
  \mathbf{A}\left(\nabla\cdot\mathbf{B}\right)
  -\mathbf{B}\left(\nabla\cdot\mathbf{A}\right)
  +\left(\mathbf{B}\cdot\nabla\right)\mathbf{A}
  -\left(\mathbf{A}\cdot\nabla\right)\mathbf{B} $
\end{itemize}

\subsubsection{Second derivatives}
\begin{itemize}
\item
  $\nabla\cdot(\nabla\times\mathbf{A})=0 $
\item
  $\nabla\times(\nabla\psi)= \mathbf{0} $
\item
  $\nabla\cdot(\nabla\psi)=\nabla^{2}\psi $
\item
  $\nabla\left(\nabla\cdot\mathbf{A}\right)
  -\nabla\times\left(\nabla\times\mathbf{A}\right)=\nabla^{2}\mathbf{A} $
\item
  $\nabla\cdot(\phi\nabla\psi)=\phi\nabla^{2}\psi + \nabla\phi\cdot\nabla\psi $
\item
  $\psi\nabla^2\phi-\phi\nabla^2\psi=
  \nabla\cdot\left(\psi\nabla\phi-\phi\nabla\psi\right)$
\item
  $\nabla^2(\phi\psi)=
  \phi\nabla^2\psi+2\nabla\phi\cdot\nabla\psi+\psi\nabla^2\phi$
\item
  $\nabla^2(\psi\mathbf{A})=
  \mathbf{A}\nabla^2\psi+2(\nabla\psi\cdot\nabla)\mathbf{A}
  +\psi\nabla^2\mathbf{A}$
\item
  $\nabla^2(\mathbf{A}\cdot\mathbf{B})=
  \mathbf{A}\cdot\nabla^2\mathbf{B} - \mathbf{B}\cdot\nabla^2\mathbf{A}
  + 2\nabla\cdot((\mathbf{B}\cdot\nabla)\mathbf{A}
  + \mathbf{B}\times\nabla\times\mathbf{A})$
\end{itemize}

\subsubsection{Third derivatives}
\begin{itemize}
\item
  $\nabla^{2}(\nabla\psi) = \nabla(\nabla\cdot(\nabla\psi))
  = \nabla(\nabla^{2}\psi)$
\item
  $\nabla^{2}(\nabla\cdot\mathbf{A})
  =\nabla\cdot(\nabla(\nabla\cdot\mathbf{A}))=\nabla\cdot(\nabla^{2}\mathbf{A})$
\item
  $\nabla^{2}(\nabla\times\mathbf{A}) =
  -\nabla\times(\nabla\times(\nabla\times\mathbf{A}))
  = \nabla\times(\nabla^{2}\mathbf{A})$
\end{itemize}

\subsection{Integration}
\subsubsection{Surface–volume integrals}
In the following surface–volume integral theorems, $V$ denotes a 3d volume with
a corresponding 2d boundary $S = \partial V$. And
$\mathbf{A}\cdot d\mathbf{S}=\mathbf{A}\cdot\mathbf{n}\, dS$.
\begin{itemize}
\item
  $\int_{\partial V} \mathbf{A}\cdot d\mathbf{S}=
  \int_V \left(\nabla\cdot\mathbf{A}\right)dV$ (Divergence theorem)
\item
  $\int_{\partial V}\psi d \mathbf{S} = \int_V \nabla \psi\, dV$
\item
  $\int_{\partial V}\left(\hat{\mathbf{n}}\times\mathbf{A}\right)dS=
  \int _{V}\left(\nabla\times\mathbf{A}\right)dV$
\item
  $\int_{\partial V}\psi\left(\nabla\varphi\cdot\hat{\mathbf{n}}\right)dS =
  \int _{V}\left(\psi\nabla^{2}\varphi+\nabla\varphi\cdot\nabla\psi\right)dV$
  (Green's first identity)
\item
  $\int_{\partial V}\left[\left(\psi\nabla\varphi
  -\varphi\nabla\psi\right)\cdot\hat{\mathbf{n}}\right]dS=
  \int_{\partial V}\left[\psi\frac{\partial\varphi}{\partial n}
    -\varphi\frac{\partial\psi}{\partial n}\right]dS
  =\int_{V}\left(\psi\nabla^{2}\varphi-\varphi\nabla^{2}\psi\right)dV$
  (Green's second identity)
\end{itemize}

\subsubsection{Curve–surface integrals}
In the following curve–surface integral theorems, $S$ denotes a 2d open surface
with a corresponding 1d boundary $C = \partial S$:
\begin{itemize}
\item
  $\int_{\partial S}\mathbf{A}\cdot d\boldsymbol{\ell}=
  \int_{S}\left(\nabla\times\mathbf{A}\right)\cdot d\mathbf{S}$
  (Stokes' theorem)
\item
  $\int_{\partial S}\psi d\boldsymbol{\ell}
  =\int_{S}\left(\hat{\mathbf{n}}\times\nabla\psi\right)dS$
\end{itemize}

\section{Spaces}
$L_2,W,H,H(\mathrm{div,curl})$, De Rham, inequality between norms, Cauchy-Schwartz...

\section{Finite elements}

\section{Equations}
\subsection{Stokes}
\subsection{Navier-Stokes}
\subsection{Heat}
\subsection{Elasticity}
\subsection{Maxwell}
The Maxwell's equations describe the generation of electric and magnetic fields
by currents, etablishing relations between the electric and magnetic fields
(resp. $\bE$ and $\bH$), the electric and magnetic flux (resp. $\bD$ and $\bB$),
the current density $\bj$ and the electric charge $\rho$. They are then
essential in the study of high field magnets.
\begin{alignat}{2}
  \nabla \times \bE &= -\frac{\partial \bB}{\partial t}
  &&\quad\text{(Faraday)} \label{eq:faraday} \\
  \nabla \times \bH &= \bj +\frac{\partial \bD}{\partial t}
  &&\quad\text{(Maxwell-Ampère)} \label{eq:maxwell-ampere} \\
  \nabla \cdot \bB &= 0
  &&\quad\text{(Gauss magnetic law)} \label{eq:gauss_magnetic_law} \\
  \nabla \cdot \bD &= \rho
  &&\quad\text{(Gauss electric law)} \label{eq:gauss_electric_law}
\end{alignat}
In static case, the time derivatives are obviously not considered. \\

These equations have to be completed with the following constitutive laws,
linking $\bB$ to $\bH$ and $\bD$ to $\bE$ from material properties.
\begin{equation}
  \label{eq:constitutive-laws}
  \bB = \mu \bH, \quad \bD = \varepsilon \bE, \quad \bj = \bj_s + \sigma \bE
\end{equation}
where $\mu$ is the magnetic permeability, $\varepsilon$ the electric permitivity
and $\bj_s$ the current source density. \\

The Gauss's magnetic law in (\ref{eq:gauss_magnetic_law}) leads to the existence
of a vectorial magnetic potential $\bA$
\begin{equation}
  \label{eq:magnetic-potential}
  \nabla\cdot\bB=0 \Rightarrow\exists\bA\text{ such that }\bB = \nabla\times\bA
\end{equation}

Combining Maxwell Ampère's equation (\ref{eq:maxwell-ampere}) with the
previously introduced constitutive laws (\ref{eq:constitutive-laws}), we have
\begin{equation}
  \nabla \times \bH = \bj
    ~\overset{ {\scriptstyle \bB = \mu \bH} }{ \Longrightarrow }
    ~\nabla \times \left( \frac{1}{\mu} \bB \right) = \bj
    ~\overset{ {\scriptstyle\bB = \nabla \times \bA} }{ \Longrightarrow }
    ~\nabla \times \left( \frac{1}{\mu} \nabla \times \bA \right) = \bj
\end{equation}

For magnetostatic problem, the independence on time removes the temporal partial
derivatives of (\ref{eq:faraday}-\ref{eq:maxwell-ampere}). Then, combining the
Maxwell-Ampère equation with (\ref{eq:constitutive-laws}) and
(\ref{eq:magnetic-potential}) gives the potential based magnetostatic problem
\begin{equation}
  \label{eq:static-potential}
  \nabla \times \left( \frac{1}{\mu} \nabla \times \bA \right) = \bj
\end{equation}

We consider here the classical boundary condition for magnetostatic, which
impose the tangential component of magnetic potential on the boundary that reads
\begin{equation}
  \label{eq:boundary-cond}
  \bA \times \mathbf{n} = \bA_D ~\text{on} ~\partial \Omega
\end{equation}
where $\mathbf{n}$ is the outward unit normal on $\delta \Omega$.
Classically $\bA_D$ is set to zero. \\

The problem (\ref{eq:static-potential}) has not a unique solution. Indeed, if
$\bA$ is solution of (\ref{eq:static-potential}),
$\bar{\bA} = \bA + \nabla \phi$ is solution for any $\phi$ since the curl of a
gradient field is always zero. We shall even so remark that the gradient field
$\nabla \phi$ doesn't affect the magnetic flux $B$ which is the classical
quantity of interest. \\

Nevertheless, the unicity of the solution is essential for the numerical solving.

The first way to guarantee the solution unicity consists in adding a condition
on the divergence, using a gauge. The use of the Coulomb gauge
$\nabla\cdot\bA=0$ adds a divergence-free condition, and the unique solution is
obtained by solving a saddle point problem. A second way consists in considering
the ungauged magnetostatic problem (\ref{eq:static-potential}) as a special case
of the potential based full maxwell problem in frequency domain.

\subsubsection{Saddle-point formulation}
\label{sec:saddle-point-mawxell}
The add of the Coulomb gauge (divergence-free condition) to the potential based
magnetostatic problem (\ref{eq:static-potential}) is managed by a scalar
Lagrange multiplier $p$, giving the following problem to solve
\begin{alignat}{2}
  \nabla \times \left( \frac{1}{\mu} \nabla \times \bA \right) + \nabla p &= \bj &&\text{ on } \Omega\\
  \nabla \cdot A &= 0 &&\text{ on } \Omega \\
  \bA \times n &= \bA_D &&\text{ on } \partial \Omega \\
  p &= 0 &&\text{ on } \partial \Omega
\end{alignat}

\paragraph{Variational formulation}
The variational formulation the consists in finding
$(\bA,p)\in(X\subset H_{\mathrm{curl}}(\Omega) \times H^1_0(\Omega))$ such that
\begin{alignat}{2}
  \int_{\Omega}\frac{1}{\mu}(\nabla \times \bA) \cdot (\nabla \times \bv)
  + \int_{\partial\Omega}\frac{1}{\mu}
  (\nabla\times\bA)\cdot(\bv\times\mathbf{n})
  + \int_{\Omega} \bv \cdot \nabla p &= \int_{\Omega} \bj \cdot \bv
  &&\quad\forall \bv \in X \label{eq:vf-saddle-point-1} \\
  \int_{\Omega} \bA \cdot \nabla q &= 0
  &&\quad\forall q \in H^1_0(\Omega) \label{eq:vf-saddle-point-2}
\end{alignat}

The Dirichlet boundary condition (\ref{eq:boundary-cond}) on $\bA$ imposed on strong form vanishes the boundary term of (\ref{eq:vf-saddle-point}) and the condition is directly taken into account in the definition of the function space $X = H_{\bA_D,\mathrm{curl}}(\Omega) = \{ v \in H_{\mathrm{curl}}(\Omega) \mid v \times \mathbf{n} = \bA_D ~\text{on} ~\delta \Omega\}$. The variational formulation then consists in finding $(\bA,p) \in ( H_{\bA_D,\mathrm{curl}}(\Omega) \times H^1_0(\Omega))$ such that
\begin{eqnarray}
  \label{eq:vf-saddle-point-strong}
  && \displaystyle{ \int_{\Omega}\frac{1}{\mu}(\nabla \times \bA) \cdot (\nabla \times \bv)
    + \int_{\delta \Omega}\frac{1}{\mu} (\nabla \times \bA) \cdot A_D
    + \int_{\Omega} \bv \cdot \nabla p = \int_{\Omega} \bj \cdot \bv} ~~\forall \bv \in X \\
  && \displaystyle{ \int_{\Omega} \bA \cdot \nabla q } = 0 ~~\forall q \in H^1_0(\Omega) \nonumber
\end{eqnarray}

We can also impose (\ref{eq:boundary-cond}) on weak form, adding symetrization and penalisation terms and then avoiding to add condition in $X$ function space, i.e. $X = H_{\mathrm{curl}}(\Omega)$. We denote $\gamma$ the penalisation coefficient, and $h_s$ the mesh size. The variational formulation then consists in finding $(\bA,p) \in ( H_{\mathrm{curl}}(\Omega) \times H^1_0(\Omega))$ such that
\begin{eqnarray}
  \label{eq:vf-saddle-point-weak}
  && \displaystyle{ \int_{\Omega}\frac{1}{\mu}(\nabla \times \bA) \cdot (\nabla \times \bv)
    + \int_{\delta \Omega}\frac{1}{\mu} (\nabla \times \bA) \cdot (\bv \times \mathbf{n})
    + \int_{\delta \Omega}\frac{1}{\mu} (\nabla \times \bv) \cdot (\bA \times \mathbf{n})
    + \int_{\delta \Omega} \frac{\gamma}{h_s} \frac{1}{\mu} (\bv \times \mathbf{n}) \cdot (\bA \times \mathbf{n}) } \nonumber \\
  &&+ \displaystyle{ \int_{\Omega} \bv \cdot \nabla p
    = \int_{\Omega} \bj \cdot \bv}
  + \int_{\delta \Omega}\frac{1}{\mu} (\nabla \times \bv) \cdot \bA_D
  + \int_{\delta \Omega} \frac{\gamma}{h_s} \frac{1}{\mu} (\bv \times \mathbf{n}) \cdot \bA_D
  ~~\forall \bv \in X \\
  && \displaystyle{ \int_{\Omega} \bA \cdot \nabla q } = 0 ~~\forall q \in H^1_0(\Omega) \nonumber
\end{eqnarray}

The vectorial magnetic potential $\bA$ is approximated using Nédélec finite elements of lowest order, and the scalar multiplier $p$ is approximated with Lagrange finite elements of order $k$.
We denote by $V_h \subset X $ and $Q_h \subset H^1_0(\Omega)$ the associated finite elements spaces, respectively. Considering $\{ \psi_j \}_{j=1}^{n}$ (resp. $\{ \phi_j \}_{i=1}^{m}$) the finite elements basis functions of $V_h$ (resp. $Q_h$), the discrete approximations $\bA_h$ of $\bA$ and $p_h$ of $p$ reads
\begin{equation}
  \label{equ:saddle-point-discretization}
  \bA_h = \sum \limits_{j=1}^{n} a_j \psi_j \qquad \text{and} \qquad p_h = \sum \limits_{i=1}^{n} p_i \phi_i
\end{equation}

Replacing $\bA$ and $p$ by their discretizations (\ref{equ:saddle-point-discretization}) in the variational formulations (\ref{eq:vf-saddle-point-strong}) or (\ref{eq:vf-saddle-point-weak}), the discrete magnetostatic saddle-point formulation yields on the form
\begin{equation}
  \label{equ:saddle-point-matricial}
  \left (
    \begin{array}{cc}
      \mathcal{A} & \mathcal{B}^{T} \\
      \mathcal{B} & 0 \\
    \end{array}
  \right)
  \left (
    \begin{array}{c}
      \bA \\
      p \\
    \end{array}
  \right)
  =
  \left(
    \begin{array}{c}
      \mathbf{f} \\
      0 \\
    \end{array}
  \right)
\end{equation}

If (\ref{eq:boundary-cond}) is imposed on the strong form as in (\ref{eq:vf-saddle-point-strong}), the matrix $\mathcal{A}$ reads
\begin{equation}
  \label{eq:saddle-point-A-strong}
  \mathcal{A}_{i,j} = \displaystyle{ \int_{\Omega} \frac{1}{\mu} (\nabla \times \psi_j) \cdot (\nabla \times \psi_i) }
\end{equation}
The enforcement of (\ref{eq:boundary-cond}) on weak form (\ref{eq:vf-saddle-point-weak}) adds symetrization and penalisation in $\mathcal{A}$ which then reads
\begin{equation}
  \label{eq:saddle-point-A-weak}
  \mathcal{A}_{i,j} = \displaystyle{ \int_{\Omega} \frac{1}{\mu} (\nabla \times \psi_j) \cdot (\nabla \times \psi_i) }
  + \displaystyle{ \int_{\delta \Omega} \frac{1}{\mu} (\nabla \times \psi_j) \cdot (\psi_i \times \mathbf{n}) }
  + \displaystyle{ \int_{\delta \Omega} \frac{1}{\mu} (\nabla \times \psi_i) \cdot (\psi_j \times \mathbf{n}) }
  + \displaystyle{ \int_{\delta \Omega} \frac{1}{\mu} \frac{\gamma}{h_s} (\psi_j \times \mathbf{n}) \cdot (\psi_i \times \mathbf{n}) }
\end{equation}

The matrix $\mathcal{B}$ of (\ref{equ:saddle-point-matricial}) reads
\begin{equation}
  \label{eq:saddle-point-B}
  \mathcal{B}_{i,j} = \displaystyle{ \int_{\Omega} \psi_j \cdot \nabla \phi_i }
\end{equation}
and the right hand side vector $\mathbf{f}$ is the discretization of the source term $\bj$ to which symetrization and penalisation terms can be added
\begin{equation}
  \mathbf{f}_i = \displaystyle{ \int_{\Omega} \bj \cdot \psi_i }
  + \displaystyle{ \int_{\delta \Omega} \frac{1}{\mu} (\nabla \psi_i) \cdot \bA_D }
  + \displaystyle{ \int_{\delta \Omega} \frac{1}{\mu} \frac{\gamma}{h_s} (\psi_i \times \mathbf{n}) \cdot \bA_D }
\end{equation}


\subsection{Table of symbols}
\begin{figure}[H]
  \centering
  \begin{tabular}{|>{$}c<{$}|c|>{$}c<{$}|>{$}c<{$}|}
    \hline
    \multicolumn{4}{|c|}{Magnetostatic} \\
    \hline
    \text{Symbol} & Quantity & \text{Unit Name} & \text{Unit SI} \\
    \hline
    \bE & Electric field & N\cdot C^{-1}=V\cdot m^{-1} & kg\cdot m\cdot s^{-3}\cdot A^{-1} \\
    \bH & Magnetic field & A\cdot m^{-1} & A\cdot m^{-1} \\
    \bD & Electric flux & C\cdot m^{-2} & A\cdot s\cdot m^{-2} \\
    \bB & Magnetic flux & \text{Tesla} (T) & kg\cdot s^{-2}\cdot A^{-1} \\
    \bj & Current density & A\cdot m^{-2} & A\cdot m^{-2} \\
    \rho & electric charge & C\cdot m^{-3} & A\cdot s\cdot m^{-3} \\
    \mu & Magnetic permeability & H\cdot m^{-1} & kg\cdot m\cdot s^{-2}\cdot A^{-2} \\
    \varepsilon & Electric permitivity & F\cdot m^{-1} & kg^{-1}\cdot m^{-3}\cdot s^{4}\cdot A^{2} \\
    \hline
  \end{tabular}
\end{figure}
\end{document}
