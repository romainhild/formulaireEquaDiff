\section{Maxwell}
The Maxwell's equations describe the generation of electric and magnetic fields
by currents, etablishing relations between the electric and magnetic fields
(resp. $\bE$ and $\bH$), the electric and magnetic flux (resp. $\bD$ and $\bB$),
the current density $\bj$ and the electric charge $\rho$. They are then
essential in the study of high field magnets.
\begin{alignat}{2}
  \nabla \times \bE &= -\frac{\partial \bB}{\partial t}
  &&\quad\text{(Faraday)} \label{eq:faraday} \\
  \nabla \times \bH &= \bj +\frac{\partial \bD}{\partial t}
  &&\quad\text{(Maxwell-Ampère)} \label{eq:maxwell-ampere} \\
  \nabla \cdot \bB &= 0
  &&\quad\text{(Gauss magnetic law)} \label{eq:gauss_magnetic_law} \\
  \nabla \cdot \bD &= \rho
  &&\quad\text{(Gauss electric law)} \label{eq:gauss_electric_law}
\end{alignat}
In static case, the time derivatives are obviously not considered. \\

These equations have to be completed with the following constitutive laws,
linking $\bB$ to $\bH$ and $\bD$ to $\bE$ from material properties.
\begin{equation}
  \label{eq:constitutive-laws}
  \bB = \mu \bH, \quad \bD = \varepsilon \bE, \quad \bj = \bj_s + \sigma \bE
\end{equation}
where $\mu$ is the magnetic permeability, $\varepsilon$ the electric permitivity
and $\bj_s$ the current source density. \\

The Gauss's magnetic law in (\ref{eq:gauss_magnetic_law}) leads to the existence
of a vectorial magnetic potential $\bA$
\begin{equation}
  \label{eq:magnetic-potential}
  \nabla\cdot\bB=0 \Rightarrow\exists\bA\text{ such that }\bB = \nabla\times\bA
\end{equation}

Combining Maxwell Ampère's equation (\ref{eq:maxwell-ampere}) with the
previously introduced constitutive laws (\ref{eq:constitutive-laws}), we have
\begin{equation}
  \nabla \times \bH = \bj
    ~\overset{ {\scriptstyle \bB = \mu \bH} }{ \Longrightarrow }
    ~\nabla \times \left( \frac{1}{\mu} \bB \right) = \bj
    ~\overset{ {\scriptstyle\bB = \nabla \times \bA} }{ \Longrightarrow }
    ~\nabla \times \left( \frac{1}{\mu} \nabla \times \bA \right) = \bj
\end{equation}

For magnetostatic problem, the independence on time removes the temporal partial
derivatives of (\ref{eq:faraday}-\ref{eq:maxwell-ampere}). Then, combining the
Maxwell-Ampère equation with (\ref{eq:constitutive-laws}) and
(\ref{eq:magnetic-potential}) gives the potential based magnetostatic problem
\begin{equation}
  \label{eq:static-potential}
  \nabla \times \left( \frac{1}{\mu} \nabla \times \bA \right) = \bj
\end{equation}

We consider here the classical boundary condition for magnetostatic, which
impose the tangential component of magnetic potential on the boundary that reads
\begin{equation}
  \label{eq:boundary-cond}
  \bA \times \mathbf{n} = \bA_D ~\text{on} ~\partial \Omega
\end{equation}
where $\mathbf{n}$ is the outward unit normal on $\partial \Omega$.
Classically $\bA_D$ is set to zero. \\

The problem (\ref{eq:static-potential}) has not a unique solution. Indeed, if
$\bA$ is solution of (\ref{eq:static-potential}),
$\bar{\bA} = \bA + \nabla \phi$ is solution for any $\phi$ since the curl of a
gradient field is always zero. We shall even so remark that the gradient field
$\nabla \phi$ doesn't affect the magnetic flux $B$ which is the classical
quantity of interest. \\

Nevertheless, the unicity of the solution is essential for the numerical solving.

The first way to guarantee the solution unicity consists in adding a condition
on the divergence, using a gauge. The use of the Coulomb gauge
$\nabla\cdot\bA=0$ adds a divergence-free condition, and the unique solution is
obtained by solving a saddle point problem. A second way consists in considering
the ungauged magnetostatic problem (\ref{eq:static-potential}) as a special case
of the potential based full maxwell problem in frequency domain.

\subsection{Saddle-point formulation}
\label{sec:saddle-mawxell}
The add of the Coulomb gauge (divergence-free condition) to the potential based
magnetostatic problem (\ref{eq:static-potential}) is managed by a scalar
Lagrange multiplier $p$, giving the following problem to solve
\begin{alignat}{2}
  \nabla \times \left( \frac{1}{\mu} \nabla \times \bA \right) + \nabla p
  &= \bj &&\text{ on } \Omega\\
  \nabla \cdot A &= 0 &&\text{ on } \Omega \\
  \bA \times \bn &= \bA_D &&\text{ on } \partial \Omega \label{eq:saddle-bc} \\
  p &= 0 &&\text{ on } \partial \Omega
\end{alignat}

\subsubsection{Variational formulation}
The variational formulation the consists in finding
$(\bA,p)\in(X\subset H_{\mathrm{curl}}(\Omega) \times H^1_0(\Omega))$ such that
\begin{alignat}{2}
  \int_{\Omega}\frac{1}{\mu}(\nabla \times \bA) \cdot (\nabla \times \bv)
  + \int_{\partial\Omega}\frac{1}{\mu}
  (\nabla\times\bA)\cdot(\bv\times\mathbf{n})
  + \int_{\Omega} \bv \cdot \nabla p &= \int_{\Omega} \bj \cdot \bv
  &&\quad\forall \bv \in X \label{eq:vf-saddle-point-1} \\
  \int_{\Omega} \bA \cdot \nabla q &= 0
  &&\quad\forall q \in H^1_0(\Omega) \label{eq:vf-saddle-point-2}
\end{alignat}

The Dirichlet boundary condition (\ref{eq:saddle-bc}) on $\bA$ imposed on strong
form vanishes the boundary term of (\ref{eq:vf-saddle-point-1}) and the
condition is directly taken into account in the definition of the function space
\[
X = H_{\bA_D,\mathrm{curl}}(\Omega)
= \{ v\in H_{\mathrm{curl}}(\Omega)
\mid v\times\mathbf{n}=\bA_D\text{ on }\partial\Omega\}
\]
The variational formulation then consists in finding
$(\bA,p) \in ( H_{\bA_D,\mathrm{curl}}(\Omega) \times H^1_0(\Omega))$ such that
\begin{alignat}{2}
\int_{\Omega}\frac{1}{\mu}(\nabla \times \bA) \cdot (\nabla \times \bv)
+ \int_{\partial \Omega}\frac{1}{\mu} (\nabla \times \bA) \cdot A_D
+ \int_{\Omega} \bv \cdot \nabla p &= \int_{\Omega} \bj \cdot \bv
&&\quad\forall \bv \in X \label{eq:vf-saddle-strong-1} \\
\int_{\Omega} \bA \cdot \nabla q &= 0
&&\quad\forall q \in H^1_0(\Omega) \nonumber%\label{eq:vf-saddle-strong-2}
\end{alignat}

We can also impose (\ref{eq:saddle-bc}) on weak form, adding symetrization and
penalisation terms and then avoiding to add condition in $X$ function space,
i.e. $X = H_{\mathrm{curl}}(\Omega)$. We denote $\gamma$ the penalisation
coefficient, and $h_s$ the mesh size. The variational formulation then consists
in finding $(\bA,p) \in ( H_{\mathrm{curl}}(\Omega) \times H^1_0(\Omega))$ such
that
\begin{alignat}{2}
  \int_{\Omega}\frac{1}{\mu}(\nabla \times \bA) \cdot (\nabla \times \bv)
  + \int_{\Omega} \bv \cdot \nabla p
  +\int_{\partial\Omega}\frac{\gamma}{h_s}\frac{1}{\mu}
  (\bv \times \bn) \cdot (\bA \times \bn)
  &&& \nonumber \\
  +\int_{\partial\Omega}\frac{1}{\mu}(\nabla\times\bA)\cdot(\bv\times\bn)
  +\int_{\partial\Omega}\frac{1}{\mu}(\nabla\times\bv)\cdot(\bA\times\bn)
  & = \int_{\Omega} \bj \cdot \bv
  + \int_{\partial \Omega}\frac{1}{\mu} (\nabla \times \bv) \cdot \bA_D
  &&\quad \forall \bv \in X \label{eq:vf-saddle-weak-1} \\
  &   + \int_{\partial\Omega}\frac{\gamma}{h_s}
  \frac{1}{\mu}(\bv\times\bn)\cdot \bA_D
  && \nonumber \\
  \int_{\Omega} \bA \cdot \nabla q &= 0
  &&\quad\forall q \in H^1_0(\Omega) \nonumber%\label{eq:vf-saddle-weak-2}
\end{alignat}

\subsubsection{Discretization}
The vectorial magnetic potential $\bA$ is approximated using Nédélec finite
elements of lowest order, and the scalar multiplier $p$ is approximated with
Lagrange finite elements of order $k$. We denote by $V_h \subset X $ and
$Q_h \subset H^1_0(\Omega)$ the associated finite elements spaces, respectively.
Considering $\{ \psi_j \}_{j=1}^{n}$ (resp. $\{ \phi_j \}_{i=1}^{m}$) the finite
elements basis functions of $V_h$ (resp. $Q_h$), the discrete approximations
$\bA_h$ of $\bA$ and $p_h$ of $p$ reads
\begin{equation}
  \label{eq:saddle-discretization}
  \bA_h = \sum \limits_{j=1}^{n} a_j \psi_j \qquad \text{and} \qquad p_h = \sum \limits_{i=1}^{n} p_i \phi_i
\end{equation}

Replacing $\bA$ and $p$ by their discretizations \ref{eq:saddle-discretization}
in the variational formulations (\ref{eq:vf-saddle-strong-1}) or
(\ref{eq:vf-saddle-weak-1}), the discrete magnetostatic saddle-point formulation
yields on the form
\begin{equation}
  \label{eq:saddle-matricial}
  \begin{pmatrix}
    \mathcal{A} & \mathcal{B}^{T} \\
    \mathcal{B} & 0 \\
  \end{pmatrix}
  \begin{pmatrix}
    \bA \\
    p \\
  \end{pmatrix}
  =
  \begin{pmatrix}
    \mathbf{f} \\
    0 \\
  \end{pmatrix}
\end{equation}

If (\ref{eq:saddle-bc}) is imposed on the strong form as in
(\ref{eq:vf-saddle-strong-1}), the matrix $\mathcal{A}$ reads
\begin{equation}
  \label{eq:saddle-A-strong}
  \mathcal{A}_{i,j} = \int_{\Omega} \frac{1}{\mu}
  (\nabla \times \psi_j) \cdot (\nabla \times \psi_i)
\end{equation}
The enforcement of (\ref{eq:saddle-bc}) on weak form
(\ref{eq:vf-saddle-weak-1}) adds symetrization and penalisation in
$\mathcal{A}$ which then reads
\begin{equation}
  \label{eq:saddle-A-weak}
  \mathcal{A}_{i,j} = \int_{\Omega} \frac{1}{\mu} (\nabla \times \psi_j)
  \cdot (\nabla \times \psi_i)
  + \int_{\partial \Omega} \frac{1}{\mu} (\nabla \times \psi_j)
  \cdot (\psi_i \times \bn)
  + \int_{\partial \Omega} \frac{1}{\mu} (\nabla \times \psi_i)
  \cdot (\psi_j \times \bn)
  + \int_{\partial \Omega} \frac{1}{\mu} \frac{\gamma}{h_s} (\psi_j \times \bn)
  \cdot (\psi_i \times \bn)
\end{equation}

The matrix $\mathcal{B}$ of (\ref{eq:saddle-matricial}) reads
\begin{equation}
  \label{eq:saddle-point-B}
  \mathcal{B}_{i,j} = \int_{\Omega} \psi_j \cdot \nabla \phi_i
\end{equation}
and the right hand side vector $\mathbf{f}$ is the discretization of the source
term $\bj$ to which symetrization and penalisation terms can be added
\begin{equation}
  \mathbf{f}_i = \displaystyle{ \int_{\Omega} \bj \cdot \psi_i }
  + \int_{\partial \Omega} \frac{1}{\mu} (\nabla \psi_i) \cdot \bA_D
  + \int_{\partial \Omega} \frac{1}{\mu}
  \frac{\gamma}{h_s} (\psi_i \times \bn) \cdot \bA_D
\end{equation}

\subsection{Regularized formulation}
\label{sec:regul-maxwell}
We consider here the Maxwell's equations
(\ref{eq:faraday}-\ref{eq:gauss_electric_law}) given on their full formulation
(with time derivatives), which can describe all electromagnetic phenomena.

The time dependency of these equation can be removed using a time Fourier
transform
\begin{equation}
  \label{eq:fourier-transform}
  \mathcal{F}[f(t)](\omega) : f(t) \longrightarrow \mathcal{F}[f(t)](\omega)
  = \int_{-\infty}^{+\infty} f(t) e^{-iwt} dt
\end{equation}

The ungauged magnetostatic problem (\ref{eq:static-potential}) is a special case
of the time harmonic Maxwell's equations with $\omega = 0$. The full maxwell
problem expressed in frequency domain is regular for all $\omega > 0$, and then
has a unique solution. The idea to regularized the magnetostatic problem
(\ref{eq:static-potential}) is based on the consideration that the
electromagnetic fields obtained at low frequencies are a good approximation of
magnetostatics fields.

From Maxwell-Ampère equation (\ref{eq:maxwell-ampere}) in
(\ref{eq:static-potential}) and (\ref{eq:constitutive-laws}), we can deduce
\begin{equation}
  \label{eq:maxwell-ampere-reg}
  \nabla \times \left( \frac{1}{\mu} \nabla \times \bA \right)
  = \bj_s + \sigma \bE + \epsilon \frac{\partial \bE}{\partial t}
\end{equation}
And from Faraday's law in (\ref{eq:static-potential}), we have :
\begin{equation}
  \label{eq:faraday-reg}
  \nabla \times \bE = -\frac{\partial (\nabla \times \bA) }{\partial t}
  ~\Rightarrow ~\bE = -\frac{\partial \bA}{\partial t}
\end{equation}

Combining (\ref{eq:maxwell-ampere-reg}) and (\ref{eq:faraday-reg}), we deduce
\begin{equation}
  \label{eq:time-potential}
  \nabla \times \left( \frac{1}{\mu} \nabla \times \bA \right)
  + \epsilon \frac{\partial^2 \bA}{\partial t^2}
  + \sigma \frac{\partial \bA}{\partial t}
  = \bj_s
\end{equation}

The Fourier transform of the derivative terms of (\ref{eq:time-potential}) are
obtained combining (\ref{eq:fourier-transform}) with a simple integration by
parts, leading to a multiplication with $i \omega$
\begin{equation*}
  \mathcal{F}[\frac{\partial \bA}{\partial t}](\omega)
  = i \omega \mathcal{F}[A](\omega)
\end{equation*}

We can then deduce the time harmonic equation
\begin{equation}
  \label{eq:time-harmonic}
  \nabla \times \left( \frac{1}{\mu} \nabla \times \bA \right)
  + (\sigma i \omega - \epsilon \omega^2) \bA = \bj_s
\end{equation}

Our operator for magnetostatic equation (\ref{eq:static-potential}) can then be
"regularized" adding to it a multiple of magnetic potential $\bA$, mimicking the
last term $(\sigma i \omega - \epsilon\omega^2) \bA$ of (\ref{eq:time-harmonic})
\begin{equation}
  \label{eq:regular-static-potential}
  \nabla \times \left( \frac{1}{\mu} \nabla \times \bA_{\epsilon} \right)
  + \alpha \bA_{\epsilon} = \bj
\end{equation}
The solution $\bA_{\epsilon}$ of the regularized problem
(\ref{eq:regular-static-potential}) converges to the solution $\bA$ of the
initial problem (\ref{eq:static-potential}) for $\epsilon \longrightarrow 0$,
and we can show that
\begin{equation}
  \label{eq:bounds-reg}
  \parallel \bA_{\epsilon} \parallel_{L_2}
  \leqslant \frac{1}{\epsilon} \parallel \bj \parallel_{L_2}
  \quad
  \text{and}
  \quad
  \exists c ~\text{such that}
  ~\parallel \bA - \bA_{\epsilon} \parallel_{H_{curl}}
  \leqslant c \parallel \bj \parallel_{L_2}
\end{equation}

\subsubsection{Variational formulation}
The variational formulation obtained from (\ref{eq:regular-static-potential})
consists in finding $\bA \in X \subset H_{\mathrm{curl}}(\Omega)$ such that
\begin{alignat}{2}
  \label{eq:vf-regul}
  \int_{\Omega}\frac{1}{\mu}(\nabla \times \bA) \cdot (\nabla \times \bv)
  + \int_{\partial \Omega}\frac{1}{\mu} (\nabla\times\bA) \cdot (\bv\times\bn )
  + \int_{\Omega}\alpha \bA \cdot \bv  &= \int_{\Omega} \bj \cdot \bv
  &&\quad \forall \bv \in X
\end{alignat}
which $X$ to be determined. \\

If the Dirichlet boundary condition (\ref{eq:boundary-cond}) on $A$ is impose on
strong form, it removes the boundary term of (\ref{eq:vf-regul})
and the condition is inherent to the function space
$X = H_{\bA_D,\mathrm{curl}}(\Omega)$. The variational formulation becomes :\\
Find $\bA \in H_{\bA_D,\mathrm{curl}}(\Omega)$ such that
\begin{alignat}{2}
  \label{eq:vf-regul-strong}
  \int_{\Omega}\frac{1}{\mu}(\nabla \times \bA) \cdot (\nabla \times \bv)
  + \int_{\Omega}\alpha \bA \cdot \bv  &= \int_{\Omega} \bj \cdot \bv
  && \quad \forall \bv \in H_{0,\mathrm{curl}}(\Omega)
\end{alignat}

We can also impose the Dirichlet boundary conditions on weak form, adding
symetrization and penalisation terms and then avoiding to add condition in $X$
function space, i.e. $X = H_{\mathrm{curl}}(\Omega)$. As previously, $\gamma$ is
the penalisation coefficient and $h_s$ the mesh size. The variational
formulation consists then in finding $\bA \in H_{\mathrm{curl}}(\Omega)$ such
that
\begin{alignat}{2}
  \int_{\Omega}\frac{1}{\mu}(\nabla \times \bA) \cdot (\nabla \times \bv)
  + \int_{\Omega}\alpha \bA \cdot \bv
  + \int_{\partial \Omega} \frac{\gamma}{h_s} \frac{1}{\mu}
  (\bv \times \bn ) \cdot (\bA \times \bn ) &&& \nonumber \\
  + \int_{\partial \Omega}\frac{1}{\mu} (\nabla \times \bA) \cdot (\bv\times\bn)
  + \int_{\partial \Omega}\frac{1}{\mu} (\nabla \times \bv) \cdot (\bA\times\bn)
  &=
  \int_{\Omega} \bj \cdot \bv
  + \int_{\partial \Omega}\frac{1}{\mu} (\nabla \times \bv) \cdot \bA_D
  && \quad \forall \bv \in H_{\mathrm{curl}}(\Omega) \label{eq:vf-regul-weak} \\
  & + \int_{\partial\Omega} \frac{\gamma}{h_s} \frac{1}{\mu}
  (\bv \times \bn) \cdot \bA_D \nonumber
\end{alignat}

\subsection{Hybrid formulation}
