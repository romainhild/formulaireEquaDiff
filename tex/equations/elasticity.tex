\section{Elasticity}

As $\Omega_T$ is at equilibrium state, the equation of motions becomes the
following equilibrium equation :
\begin{equation}
  \label{eq:equilibrium}
  \nabla\cdot\bbar{\sigma} + \bff = 0
\end{equation}

where :
\begin{itemize}
  \item $\bbar{\sigma}$ is the stress tensor
  \item $\bar{f}$ represents the volume forces applied on $\Omega$
\end{itemize}

The quantity we focus on is the displacement vector $\bu$, which doesn't appear
in the equilibrium equation (\ref{eq:equilibrium}). \\
We have to introduce the tensor of small deformations $\bbar{\varepsilon}$ :
\begin{equation}
\label{eq:small-deformation}
\bbar{\varepsilon} = \frac{1}{2}(\nabla \bu + \nabla \bu^T)
\end{equation}
And the Hooke's law allows to link stress tensor $\bbar{\sigma}$ with tensor of
small deformation $\bbar{\varepsilon}$ :
\begin{equation}
  \label{eq:hooke}
  \bbar{\sigma}(\bbar{\varepsilon}) = \frac{E}{1+\nu} \left(\bbar{\varepsilon}
  + \frac{\nu}{1-2\nu} Tr(\bbar{\varepsilon}) I \right)
\end{equation}
where :
\begin{itemize}
 \item $E$ is Young modulus
 \item $\nu$ is Poisson's ratio
 \item $I$ is the identity tensor
\end{itemize}
We introduce the $\lambda$ and $\mu$ the Lamé coefficients :
\begin{equation*}
  \lambda=\frac{E\nu}{(1+\nu)(1-2\nu)}
  \quad\quad\text{ and }\quad\quad
  \mu=\frac{E}{2(1+\nu)}
\end{equation*}
which allow us to rewrite (\ref{eq:hooke}) as:
\begin{equation*}
  \bbar{\sigma}(\bbar{\varepsilon})
  = 2\mu\bbar{\varepsilon} + \lambda Tr(\bbar{\varepsilon})I
\end{equation*}
For simplicity, we'll omit the dependance in the follwing.\\

We need to complete the system with some boundary conditions:
\begin{empheq}[left=\empheqlbrace]{alignat=2}
  \bu &= \bu_D &&\quad \text{ on } \partial\Omega_D \label{eq:ela-bc-1} \\
  \bbar{\sigma}\cdot\bn &= \bg
  &&\quad \text{ on }\partial\Omega_P \label{eq-ela-bc-2}
\end{empheq}
where $\partial\Omega=\partial\Omega_D\cup\partial\Omega_P$.

\subsection{Variational formulation}
The variational formulation consists in finding $\bu$ in $X$ such that:
\begin{alignat}{2}
  \int_\Omega \bbar{\sigma}:\nabla\bm{\varphi}
  - \int_{\partial\Omega} \bbar{\sigma}\cdot\bn\cdot\bm{\varphi}
  & = - \int_\Omega \bff\cdot\bm{\varphi}
  && \quad \forall \bm{\varphi} \in X \label{eq:ela-vf}
\end{alignat}

We now need to make appear the displacement vector $\bu$ thanks to the Hooke's
law (\ref{eq:hooke}) in the first component of (\ref{eq:ela-vf}):
\begin{equation}
  \int_\Omega \bbar{\sigma}:\nabla\bm{\varphi}
  = 2\mu \int_\Omega \bbar{\varepsilon}:\nabla\bm{\varphi}
  + \lambda \int_\Omega Tr(\bbar{\varepsilon})I:\nabla\bm{\varphi}
\end{equation}
We have the following identities:
\begin{equation}
  Tr(\bbar{\varepsilon}) = \nabla\cdot\bu,
  \quad I:\nabla\bm{\varphi} = \nabla\cdot\bm{\varphi},
  \quad \bbar{\varepsilon}:\nabla\bm{\varphi} = \bbar{\varepsilon}:\bbar{s}
  \quad\text{with}\quad
  \bbar{s} = \frac{1}{2}(\nabla\bm{\varphi}+\nabla\bm{\varphi}^T)
\end{equation}
Which gives us the formulation:
\begin{equation}
  2\mu \int_\Omega \bbar{\varepsilon}:\bbar{s}
  + \lambda \int_\Omega (\nabla\cdot\bu)(\nabla\cdot\bm{\varphi})
  - \int_{\partial\Omega} \bbar{\sigma}\cdot\bn\cdot\bm{\varphi}
  = - \int_\Omega \bff\cdot\bm{\varphi}
  \quad \forall \bm{\varphi} \in X \label{eq:ela-vf-displa}
\end{equation}

Imposing the boundary conditions (\ref{eq:ela-bc-1}) in a strong form leads to
finding $\bu\in H^1_{\bu_D,\Omega_D}(\Omega)$ such that:
\begin{alignat}{2}
  2\mu \int_\Omega \bbar{\varepsilon}:\bbar{s}
  + \lambda \int_\Omega (\nabla\cdot\bu)(\nabla\cdot\bm{\varphi})
  & = - \int_\Omega \bff\cdot\bm{\varphi}
  + \int_{\partial\Omega_P} \bg\cdot\bm{\varphi}
  && \quad \forall \bm{\varphi} \in H^1_{0,\Omega_D}(\Omega)
  \label{eq:ela-vf-strong}
\end{alignat}

We also can impose the boundary condition (\ref{eq:ela-bc-1}) in a weak way by
adding penalty and consistance terms, giving us the following formulation:
Find $\bu\in H^1(\Omega)$ such that:
\begin{alignat}{2}
  2\mu \int_\Omega \bbar{\varepsilon}:\bbar{s}
  + \lambda \int_\Omega (\nabla\cdot\bu)(\nabla\cdot\bm{\varphi})
  + \int_{\partial \Omega_D} \frac{\gamma}{h_s} \bu \cdot \bm{\varphi}
  &&& \nonumber \\
  - \int_{\partial \Omega_D}
  (\bbar{\sigma}(\bbar{\varepsilon}) \cdot \bn) \cdot \bm{\varphi}
  - \int_{\partial \Omega_D} (\bbar{\sigma}(\bbar{s}) \cdot \bn) \cdot \bu
  &= - \int_\Omega \bff\cdot\bm{\varphi}
  + \int_{\partial\Omega_P} \bg\cdot\bm{\varphi}
  && \quad \forall \bm{\varphi}\in H^1(\Omega) \label{eq:ela-vf-weak} \\
  & + \int_{\partial \Omega_D} \frac{\gamma}{h_s} \bu_D \cdot \bm{\varphi}
  - \int_{\partial \Omega_D} (\bbar{\sigma}(\bbar{s}) \cdot \bn) \cdot \bu_D
  \nonumber
\end{alignat}

\subsection{Thermal dilation}
We can add a thermal dilatation term to the stress tensor $\bbar{\sigma}$:
\begin{equation}
  \bbar{\sigma}(\bbar{\varepsilon})
  = \bbar{\sigma}_E(\bbar{\varepsilon}) + \bbar{\sigma}_T
  = 2\mu\bbar{\varepsilon} + \lambda Tr(\bbar{\varepsilon})I
  - \frac{E}{1-2\nu}\alpha(T-T_0)I
\end{equation}

In the case where we impose the Dirichlet conditions in a strong way, we need to
add the following term to the right hand side:
\begin{equation}
  \int_\Omega \frac{E\alpha}{1-2\nu}(T-T_0)\nabla\cdot\bm{\varphi}
\end{equation}
And for the weak form of the Dirichlet conditions, add the previous term and the
following to the right hand side:
\begin{equation}
  -\int_{\partial\Omega_D} \bbar{\sigma}_T\cdot\bn\cdot\bu_D
\end{equation}
as the two following terms to the matrix:
\begin{equation}
  -\int_{\partial\Omega_D} \bbar{\sigma}_T\cdot\bn\cdot\bm{\varphi}
  -\int_{\partial\Omega_D} \bbar{\sigma}_T\cdot\bn\cdot\bu
\end{equation}
